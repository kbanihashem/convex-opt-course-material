فرض کنید که می‌خواهیم بررسی کنیم که آیا $n$ نفر جا می‌شوند یا نه. باید مساله‌ی زیر را بررسی کنیم که \lr{feasible} است یا نه.
\[
0 \le x_i \le l, \quad 0 \le y_i \le w\]\[
||x_i - x_j, y_i - y_j||_{p} \ge 2\quad i < j
\]
حال دقت کنید که 
\[
||a, b||_{1} \ge k \iff 
a + b \ge k \lor a - b\ge k \lor -a + b \ge k \lor -a - b \ge k
\]
و به طور مشابه
\[
||a, b||_{\infty} \ge k \iff 
a \ge k \lor b \ge k \lor -a \ge k \lor -b \ge k
\]
نکته‌ی مهم این است که اگر علامت $a$ و $b$ را بدانیم، می‌توانیم قید بالا را به فرم آفین بنویسیم زیرا برای این که بررسی کنیم که آیا حداقل یکی از عبارات بزرگتر یا مساوی $k$ است، کافیست آن عبارتی که در آن علامت‌ها مثبتند را بررسی کنیم چون بیشترین است.\\
با این توصیفات، برای حل مساله‌ی
\lr{feasiblity}
برای $n$ نفر، فرض می‌کنیم که بر اساس $x$ مرتب‌اند به طوری که 
$x_i \le x_{i + 1}$.
برای بررسی ترتیب $y$‌ها هم، همه‌ی
$n!$
حالت ممکن را در نظر می‌گیریم. یعنی روی همه‌ی این حالات فور می‌زنیم. در هر کدام از این حالات، چون علامت را می‌دانیم، می‌توانیم برنامه‌ی محدب را تشکیل دهیم و بررسی کنیم که آیا $n$ نفر جا می‌شوند یا خیر. حال اگر جواب نهایی $T$ باشد، می‌توانیم با شروع از 1 و امتحان همه‌ی اعداد تا زمانی که به $T$ برسیم فور بزنیم 
\footnote{
	چون حل مساله برای $T$ به مراتب سخت‌تر از حل آن برای $T- 1$ است، باینری سرچ زدن کار اشتباهی است.
}
پس در کل
\[
\sum_{i=1}^{T}i! 
\]
تا باید برنامه حل کنیم. عبارت فوق از نظر مرتبه عملا $T!$ است زیرا
\[
\sum_{i=1}^{T}i! = \sum_{i=1}^{T-1}i! + T! \le (T-1)\times (T-1)! + T! \le 2T!
\]
\textbf{کران فوق برای هر دو بخش الف و ب برقرار است.}
در صورت نیاز، می‌توان کران‌هایی بر حسب $l, w$ هم داد. به طور دقیق‌تر، در حالتی که $p=1$، اگر حول هر نفر و به مرکزیت آن یک دایره‌ی
 $l_{\infty}$
 به شعاع ۱ بزنیم (در واقع همان مربع خودمان هستند)، این دوایر همدیگر را قطع نمی‌کنند زیرا اگر قطع کنند یعنی دو نقطه نزدیک‌تر از ۲ فاصله دارند. همه‌ی دوایر در مستطیل به طول و عرض
 $l+2, w+2$
 جا می‌شوند و همچنین مساحت هرکدام حداقل ۴ است. در نتیجه 
 \[
p = \infty \implies 4T \le (l+2)(w+2) \implies T \le \frac{(l+2)(w+2)}{4}
 \]
برای نمر $l1$ هم به طور مشابه همه‌چیز در همان مستطیل جا می‌شوند اما مربع‌ها مساحتشان نصف شده پس کران بالا به شکل
\[
T \le \frac{(l+2)(w+2)}{2}
\]
در می‌آید.
کد این را من نزدم چون فکر کنم در حالتی که در هر ثانیه ۱۰ به توان ۹ تا برنامه حل شوند، ۷۷ سال طول می‌کشد که تمام شود!