\begin{enumerate}
	\item 
صورت حکم چیزی به فرم
فلان اگر و فقط اگر بهمان است. من ثابت می‌کنم که بهمان اگر و تنها اگر فلان. دقت کنید که همان مطلب ثابت شده است و صرفا چون تشکیل دوگان از این سمت برایم راحت‌تر بود این کار را می‌کنم.\\
مساله‌ی زیر را در نظر بگیرید
\[
minmize_{\mu} \quad -c^T\mu \]\[
s.t. \quad S^T\mu \le 0
\]
این مساله به وضوح محدب است (اصلا آفین است). همچنین شرط
\lr{slater}
برقرار است زیرا اگر قرار دهیم
$\mu = 0$،
در نامساوی‌ها (که همه خطی‌اند) صدق می‌کند. در نتیجه 
\lr{strong dualtiy}
برایش برقرار است. از طرفی دقت کنید که جواب این مساله بیشتر مساوی ۰ است، اگر و تنها اگر 
بخش بهمان سوال برقرار باشد یعنی برای هر 
$S^T\mu \le 0$
بتوان گفت که 
$c^T\mu \le 0$.
همچنین همواره کمتر‌مساوی ۰ است زیرا می‌توان قرار داد
$\mu = 0$.
پس جواب این مساله ۰ است اگر و تنها اگر بخش بهمان برقرار باشد.
از طرفی با تشکیل دوگان داریم
\[
\mathcal{L} = -c^T\mu + v^TS^T\mu = (-c + Sv)^T\mu
\]
که با اینفیمم‌گیری روی $\mu$، منفی بی‌نهایت است اگر ضریب ناصفر باشد و در غیر این صورت ۰ است. پس مساله‌ی دوگان به فرم زیر است
\[
maximize \quad 0 \]\[
s.t. \quad Sv - c = 0 \land v \ge 0
\]
از طرفی جواب این مساله برابر با ۰ است، اگر و تنها اگر \lr{feasible} باشد که معادل است با این که بخش فلان مساله درست باشد یعنی
$c$
یک بردار 
\lr{thermodinamically feasible}
باشد. پس حکم ثابت شد چون ثابت کردیم که فلان معادل است با ۰ بودن جواب مساله‌ی دوگان که معادل است با 
%(این بخش همواره برقرار است یعنی کلا با \lr{slater} ثابت کردم و خیلی ربط خاصی به مساله نداشت)
%(طبق \lr{slater})
 ۰ بودن مساله‌ی اصلی که معادل است با بهمان.
\item 
فرض کنید که حکم غلط باشد. مساله‌ی زیر را در نظر بگیرید
\[
minimize \quad -t\]\[
s.t. \quad S^Tm = 0 \land m \ge t
\]
در این صورت جواب این مساله، قطعا مقداری منفی‌ای نخواهد بود زیرا اگر منفی باشد، $t$ مثبت است و در نتیجه 
$m > 0$
و همچنین
$m^TS = 0$
که حکم است و در نتیجه با فرض خلف در تناقض است. پس جواب مساله بزرگتر مساوی ۰ است. از طرفی اگر قرار دهیم
$m = 0 \land t = 0$،
به وضوح به ۰ می‌رسیم. پس جواب این مساله ۰ است.\\
مساله به وضوح محدب است (مجددا همه‌چی آفین است) و شرط 
\lr{slater}
برقرار است چون می‌توانیم قرار دهیم
$t = -1, m = 0$
(البته اگر $t=0$ قرار می‌دادیم هم اوکی بود). در نتیجه 
\lr{strong dualtiy}
داریم. مساله‌ی دوگان را تشکیل می دهیم.
\[
\mathcal{L} = -t + u^T(S^Tm) + \lambda^T(t1 - m) = 
t(\lambda^T1 - 1) + (Su - \lambda)^Tm 
\]
که با اینفیمم‌گیری روی $t, m$، برابر با منفی بی‌نهایت است اگر یکی از ضرایب ناصفر باشد و در غیر این صورت ۰ است. پس مساله‌ی دوگان به فرم زیر است.
\[
maximize \quad 0\]\[
s.t. \quad \lambda \ge 0 \land Su = \lambda \land \sum 1^T\lambda = 1
\]
چون جواب مساله‌ی اصلی ۰ بود، جواب این مساله هم ۰ است و در نتیجه
\lr{feasible}
است.
پس چون داریم
\[
Su = \lambda \ge 0
\]
طبق فرض
\[
Su = 0 \implies \lambda = 0 \implies 1 = 1^T\lambda = 0 
\]
که به وضوح تناقض است.
%\item 
%مساله‌ی زیر را در نظر بگیرید
%\[
%minimize_{v, t} \quad -t\]\[
%s.t. \quad Sv \ge t
%\]
%به وضوح محدب است چون همه‌چی آفین است. همچنین شرط
%\lr{slater}
%را دارد چون کافیست $v$ دلخواه بگیریم و 
%$t$
%را برابر با کوچکترین درایه‌ی
%$Sv$
%قرار دهیم. پس 
%\lr{strong dualtiy}
%برقرار است. از طرفی طبق فرض، جواب مساله حداکثر ۰ است زیرا اگر کمتر از ۰ باشد، 
%یک بردار $v$ داریم که 
%$Sv \ge 0
%$ اما
%$Sv \ne 0$.
%از طرفی خود ۰ هم قابل دست یافتن است زیرا کافیست قرار دهیم
%$v = t = 0$.
%پس جواب مساله ۰ است. حال دوگانش را تشکیل می‌دهیم.\\
%\[
%\mathcal{L} = -t + \lambda^T(t1 - Sv) = t(-1 + \lambda^T1) - \lambda^TSv
%\]
%که با اینفیمم‌گیری روی $v, t$، مقدار لاگرانژین منفی بی‌نهایت است اگر ضرایب حداقل یکیشان ناصفر باشد و در غیر این صورت ۰ است. پس مساله‌ی دوگان به فرم زیر است.
%\[
%maximize \quad 0\]\[
%s.t. \quad
%\lambda^T1 = 1 \land \lambda^TS = 0 \land \lambda^T \ge 0
%\]
%حال دقت کنید که 
\end{enumerate}