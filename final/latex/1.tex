دقت کنید که یک مجموعه نامتناهی است اگر و تنها اگر 
ماکسیمم قدر مطلق درایه‌هایش نامتناهی باشد. این عملا بدیهی است زیرا اگر تعریف نامتناهی را نامحدود بودن نرم $l2$ بگیریم، چون در فضای $n$بعدی، نرم $l2$ بزرگتر مساوی ماکسیمم درایه است  (طبق فیثاغورث یا حتی نامساوی مثلث) و همچنین از
$\sqrt{n}$
برابر نرم $l_\infty$ کمتر است (طبق فیثاغورث)، می‌توان این فرض را کرد.\\
در نتیجه نامتناهی بودن، معادل با این است که حداقل در یکی از $2n$ جهت مختصات (همه مثبت و هم منفی در نظر گرفته شده‌اند) تصویر مجموعه نامتناهی باشد. معادلا می‌توان برای هر $2n$ حالت از بردار 
$v=(-1)^{k}e_i$،
سعی کنیم که مقدار را بیشینه کنیم. یعنی می‌توان این $2n$ برنامه را حل کرد و دید که آیا حداقل یکیشان نامتناهی است یا نه
\[
maximize v^Tx\]\[
s.t. \quad Ax \le b \land Cx = d
\]
اما چون گفته شده ۱ برنامه، همه‌ی این‌ها را با هم قاطی می‌کنیم. یعنی به شکل زیر (در زیر، $V$ همان مجموعه‌ی $2n$ برداری است که از در نظر گرفتن خود بردار‌های مختصات و منفی‌شان به دست می‌آید.)
\[
maximize \sum_{v \in V} v^Tx_{v}\]\[
Ax_{v} \le b \land Cx_{v} = d
\]
دقت کنید که برای هر کدام از $2n$ حالت مختلف، یک متغیر $n$‌تایی جدید گرفته‌ایم. ادعا می‌کنیم مثبت بی‌نهایت بودن جواب این برنامه معادل با نامتناهی بودن 
\lr{polyhedron}
است. اگر 
\lr{polyhedron}
نامتناهی باشد، یکی از مختصات‌ها به مثبت بی‌نهایت میل می‌کند. در نتیجه 
$x_{v}$
متناظر با همه‌ی دیگر را یک نقطه‌ی ثابت می‌گیریم و در نتیجه در تابع هدف، این جملات ثابت می‌مانند. 
$x_{v}$
مربوط به این مختصات اما چون می‌تواند به بی‌نهایت میل کند، $lp$ هم می‌تواند به مثبت بی‌نهایت میل کند.\\
با توجه به این که این راه حل رو میشه تو $cvx$ زد طوری که کار کنه، سراغ این که دوگانش رو دربیاریم و یه جوری ربط بدیم به این که این نامتناهی اگر و تنها اگر اون 
\lr{feasible}
باشه و اینا نمی‌رم. \\
در صورت نیاز، می‌توان برای نوشتن به فرم مناسب، ماتریس
$A'$
را در نظر گرفت که ماتریسی است که تعداد سطر‌ها و ستون‌هایش 
$2n$
 برابر $A$ است و روی قطر آن $A$ به تعداد 
 $2n$
  بار تکرار شده است. همچنین می‌توان همین کار را برای ماتریس $C'$ انجام داد. بردار‌های $d', b'$ هم صرفا $d, b$ هستند که $2n$ بار تکرار شده‌اند. بردار $k$ هم برداری است از ابعاد
$2n$
برابر $x$ که در آن $e_i$‌ها و $-e_i$‌ها پشت سر هم چیده شده‌اند. در این صورت برنامه به فرم زیر است.
\[
maximize \quad k^Tx'\]\[
s.t. A'x' \le b', \quad C'x' = d'
\]

%همچنین چون با همین حالت میشه تو $cvx$ زد، سراغ این که فرم بلوکی ماتریس $A'$ جدید که گنده‌تر شده رو بنویسم هم نمی‌رم.\\
ابعاد
$lp$
اما خب به وضوح خیلی خوب نیست و اگر $n$ تا $lp$ جدا می‌زدیم فکر کنم بهتر میشد چون عملا من هم همین کار رو کردم و در این حالت صرفا امید به $cvx$ و ایناست که خودشون ببینن که $A$ جدید اسپارسه.\\
ایده‌هایی واسه سریع‌تر کردن: 
اول یه توضیح بدم این که این راه حل، ممکنه اشتباه باشه ولی خب فکر کنم درسته (الان ساعت ۱۰ و ۴۰ه که دارم اینو می‌نویسم) و در هر حال هم برای حلش، یه 
\lr{preprocess}‌طور
نیازه که توش بیایم چک کنیم که آیا یه بردار پیدا میشه که هم تو نال‌اسپیس $C$ باشه و هم نال‌اسپیس $A$. این یه جورایی حالت خاص مساله است ولی خب نتونستم با $lp$ حلش کنم. اگه با $lp$ بشه حلش کرد میشه $lp$‌ها رو ترکیب کرد احتمالا. اما خب اگه نشه هم الگوریتم‌های دیگه‌ای که هستن وجود دارن و در هر کاربرد
\lr{practical}
اصولا نباید مهم باشه (با فرم‌های دیگه‌ی
\lr{convex}
هم ایده‌ای ذهنم نرسید. در واقع مشکل اصلی اینه که شرط $x \ne 0$ رو نمی‌دونم باید چجوری چپوند تو برنامه‌ریزی خطی). با این توضیحات، باز هم راه حل رو آوردم به این امید که اگه راه حل اصلی نمره‌ی کامل رو نگرفت، این راه حل بخشی از نمره رو جبران کنه.\\
راه حل:\\
\\این رو وقت نکردم خیلی دقیق کنم و در هر حال یه مشکل ریز داره اما خب. فکر کنم بشه گفت که اگه نامتناهی باشه، یکی از ۲ تا حالت می‌تونه پیش بیاد. یکی این که تو جهت یکی از 
$Ax - b$‌ها
و فکر کنم بشه گفت که تو این حالت، باید یه $x'$ای باشه که 
$Ax' \le 0, Ax' \ne 0, Cx = 0$.
این شرط به نظرم کافیه (در کنار
\lr{feasibillity}
طبیعتا
).
این که بگیم این شرط لازمه رو فکر کنم بشه با 
\lr{duality}
درآورد ولی خب مجددا وقت نکردم دقیق کنم.
چک کردن این هم ساده است، کافیه به جای شرط دوم بذاریم
\[
1^TAx' = -1
\]
به خاطر این که طبق شرط اول، اگر این شرط برقرار نباشه هم با اسکیل کردن می‌تونیم درستش کنیم.\\
یه حالت دیگه‌ای هم که ممکنه رخ بده اینه که تو یه جهت عمود بر سطر‌های $A, C$ نامتناهی بشه. چک کردن این رو با $lp$ هر کاری کردم نتونستم. یعنی یه شرط از جنس $y\ne0$ داره که نتونستم کاریش کنم. ولی خب میشه $A ,C$ رو گذاشت زیر هم، بعد بیایم الگوریتم‌هایی که واسه پیدا کردن 
\lr{nullspace}
وجود دارند رو بزنیم. دقت کنید که این همون حالت خاص مساله واسه وقتیه که $A$ نداریم کلا! ولی خب این حالت رو هم با $lp$ نتونستم کاریش کنم و الگوریتم جدا می‌خواد. 
\textbf{میشه ولی بهش به شکل 
\lr{preprocess}
قبل
\lr{lp}
نگاه کرد.
}
 اما خب اگه بشه این حالت رو با $lp$ حد کرد، ترکیبش با مساله کاری نداره.\\
یک نکته‌ای هم که کلا هست، اینایی که گفتم چند تا $lp$ میشن (مثلا دو سه تا) ولی خب میشه مثل همون کاری که اول مساله گفتم اوم همشون رو تبدیل کرد به یه مساله. یعنی یه سری متغیر مستقل می‌گیریم و اینا.
تنها چیزی که می‌مونه، اینه که اون حرفم رو دقیق کنم. برای اثباتش هم دقت کنید که اگر یه چیزی باشه که $Ax=0, Cx=0$، که خب نامتناهیه مجموعه جواب (کلا یه $lp$ جدا واسه چک کردن این که ناتهی باشه مجموعه جواب می‌خوام. دقت کنید که وقتی می‌گم یه چیزی باشه، طبیعتا منظورم یه چیز غیر۰ه و واسه همین هم نتونستم با $lp$ حلش کنم.) اگر همچین چیزی نباشه، یعنی این که کل جهت‌های فضا رو بردار‌های
$A, C$
پوشش می‌دند. در نتیجه برای این که نامتناهی بشه، باید ضرب داخلیش با حداقل یکی از این‌ها به مثبت یا منفی بی‌نهایت میل کنه.
ضربداخلیش با سطرای $C$ که خب همیشه ۰اه. پس باید با یکی از سطرای $A$ میل کنه به منفی بی‌نهایت (مثبتش نمیشه چون تو \lr{polyhedron} نیست). اگه یه $x'$ای باشه که ضربداخلیش با سطرای $A$ ناصفر باشه که خب همونطور که گفتم اوکیه (با $C$ هم باید ۰ باشه البته طبیعتا). پس برای این که مساله تموم شه باید صرفا بگم که این شرط لازم هم هست. برای این کار فرض کنید که همچین برداری پیدا نشه. یه بردار ثابت توی
\lr{polyhedron}
وردارید. بعد یک بردار خیلی خیلی دور هم بیاید وردارید. اختلافشون رو حساب کنید. ضرب داخلی این بردار با سطرای $A$ رو اسمش رو بذارید $s$. الان $s$ تو یکی از درایه‌هاش یه عدد منفی خیلی خیلی گنده است. تو حداقل یکی از درایه‌هاش هم ولی مثبته چون فرض کردیم که بردار خوبی که ضرب داخلیش با همه‌ی سطرا بشه کمتر مساوی ۰ و ... نداریم. الان بیاید $s$ رو نرمالایز کنید ولی. اون بخشش که مثبته، خیلی کوچیک می‌شه. به طور دقیق‌تر، چون هر چه اون نقطه‌ی دور رو دورتر بگیرم، می‌تونم نرم رو بیشتر کنم. از طرفی دقت کنید که قبل از نرمالایز کردن، کران بالا هست روی این که چقدر هر کدوم از درایه‌های $s$ می‌تونه مثبت باشه به خاطر این که الان اون نقطه‌ی ثابتی که گرفتیم (نزدیکه)، بالاخره یه ضرب داخلی‌ای داره با هر کدوم از سطرا و از اختلاف
$b_i$
با اون ضرب داخلی‌ اگه بیشتر بخوایم افزایش بدیم، خب دیگه تو 
\lr{polyhedron}
نمی‌مونه. پس شما من می‌تونم روی گوی واحد، یه بردار پیدا کنم که 
$Cx'=0$
و
$Ax' \le t$
که $t$ رو می‌تونم به دلخواه کوچک کنم (البته طبیعتا مثبته ). در نتیجه می‌تونم یه $x'$ پیدا کنم که 
$Ax' \le 0$
و
$Cx' \le 0$.
دقت کنید که چون یکی از درایه‌های
$Ax'$
همینجوریش هم یه عدد منفی اکید بود، مشکلی سر اون شرط دوم هم پیش نمیاد.\\
پس این راه هم جواب میده و سریعتره اما خب یه 
\lr{preprocess}
می‌خواد اما در هر حال  اصولا هر راه حلی که واسه حالت خاصی که
$A$
نداریم رو بدین، اصولا میشه به‌جای 
\lr{preprocess}
از اون استفاده کرد و در نتیجه اگه اون حالت خاص رو بشه با
\lr{lp}
حل کرد، این حالت کلی رو هم میشه.
